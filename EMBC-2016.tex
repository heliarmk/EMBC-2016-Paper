%
%  This is an example LaTeX file. The percent sign is used to mark the
% start of a comment.
%
%  - Michael Weeks, January, 2003
%
\documentclass[conference]{IEEEconf}
\usepackage[dvips]{graphics}
\usepackage{tikz}
\usepackage{dtklogos}
\usepackage[utf8]{inputenc}
\usetikzlibrary{mindmap}
\usepackage[hidelinks,pdfencoding=auto]{hyperref}
% Information boxes
\newcommand*{\info}[4][16.3]{%
  \node [ annotation, #3, scale=0.65, text width = #1em,
          inner sep = 2mm ] at (#2) {%
  \list{$\bullet$}{\topsep=0pt\itemsep=0pt\parsep=0pt
    \parskip=0pt\labelwidth=8pt\leftmargin=8pt
    \itemindent=0pt\labelsep=2pt}%
    #4
  \endlist
  };
}

\IEEEoverridecommandlockouts % Don't forget this command!

\begin{document}
  \title{Data Driven Approach For Transfer Data  From ANT+ Sensors}

  \author{{Petr Je\v{z}ek$^{1}$, Roman Mou\v{c}ek$^{1}$}
\thanks{$^{1}$Department of Computer Science and Engineering
New Technologies for the Information Society
Faculty of Applied Sciences
University of West Bohemia
Plzen, Czech Republic
        {\tt\small {jezekp, moucek}@kiv.zcu.cz}}%
\thanks{*Acknowledgements will be added}% <-this % stops a space
}
\maketitle


\begin{abstract}
This is the abstract. You can use this file to start your own LaTeX file,
and just delete the stuff you do not need. \LaTeX  is a lot like working
with HTML: you can specify where text effects begin, and where they end.
\end{abstract}

\section{Introduction}\label{sec:intro}
Managing diseases and health issues is worldwide still more expensive especially with aging population. For instance there are around 23 million people affected with heart failure \cite{bui2011epidemiology}. These people have been treated in hospitals for a long time. Nowadays, the situation is changing because relatively cheap solutions for a home treatment is coming to the market \cite{4761985, 5333913}. These solutions enable to make a particular shift of patients from hospitals to their homes. It brings advantages in better comfort for patients and cheapens treatment itself.  These solutions usually use a set of sensors for monitoring of a health level. Wearable sensors are usually powered from batteries. They have to operate for a long time period without possibility to change batteries frequently. As a solution new protocols with the objective of low energy consumption as ZigBee \cite{Farahani:2008:ZWN:1457417}, Bluetooth Low Energy \cite{heydon2012bluetooth} or ANT \cite{zaloker2014ant} has been defined.  Data from these sensors are transfered to remote servers where are they are processed and results are visualized to the user. Sensors usually measure a large collection of body parameters. Integration of these sensors creates a Body Area Networks (BAN). When the number of sensors connected to BAN is increasing a management and a long term storage and sustainability of data is a major requirement in future applications.

While low energy consumption standards for data transfer exist they are still too fragmented to enable an easy manipulation with obtained data. As a solution this paper presents an approach of using a unified HDF5 format for encapsulating ANT+ sensor data and its transfer to a remote storage. 

The paper is organized as follows. Section \ref{sec:state-of-the-art} describes current approaches in the domain then Section \ref{sec:ant-plus-profiles} describes existing ANT+ profiles and selects most suitable profiles for eHealth domain. Section \ref{sec:framework} describes proposed framework that facilitates the conversion of sensors data to an output HDF5 format. Then Section \ref{sec:use-case} demonstrates the usage of proposed transformation. Last Section \ref{sec:future-work} summarizes the current work and provide outlook to the future.

\section{State of The Art}\label{sec:state-of-the-art}

An approach presented in \cite{mehmood2014ontology} defines a three layers ontology describing data from different sensors. This ontology can facilitates programmer development of tools for processing sensors data.


\section{ANT+ Profiles}\label{sec:ant-plus-profiles}

\section{Proposed Framework}\label{sec:framework}

\section{Use Case}\label{sec:use-case}

\section{Conclusions and Future Work}\label{sec:future-work}


\begin{figure*}[ht]
\centering\includegraphics[width=12cm, height=10cm]{AntPlusProfiles}
\caption{\label{AntPlus}ANT+ Profiles Network}

\end{figure*}





% Now here is the reference section.

\bibliographystyle{IEEEtran}
% argument is your BibTeX string definitions and bibliography database(s)
\bibliography{EMBC-2016}
\end{document}