%
%  This is an example LaTeX file. The percent sign is used to mark the
% start of a comment.
%
%  - Michael Weeks, January, 2003
%
\documentclass[conference]{IEEEtran}
\usepackage[dvips]{graphics}
\usepackage{tikz}
\usepackage{dtklogos}
\usepackage[utf8]{inputenc}
\usetikzlibrary{mindmap}
\usepackage[hidelinks,pdfencoding=auto]{hyperref}
\usepackage{dirtree}
% Information boxes
\newcommand*{\info}[4][16.3]{%
  \node [ annotation, #3, scale=0.65, text width = #1em,
          inner sep = 2mm ] at (#2) {%
  \list{$\bullet$}{\topsep=0pt\itemsep=0pt\parsep=0pt
    \parskip=0pt\labelwidth=8pt\leftmargin=8pt
    \itemindent=0pt\labelsep=2pt}%
    #4
  \endlist
  };
}


\IEEEoverridecommandlockouts % Don't forget this command!

\begin{document}
  \title{Data Driven Approach For Transferring ANT+ Sensor Data}

  \author{{Petr Je\v{z}ek$^{1}$ Roman Mou\v{c}ek$^{1}$}
\thanks{$^{1}$Department of Computer Science and Engineering
New Technologies for the Information Society
Faculty of Applied Sciences
University of West Bohemia
Plzen, Czech Republic
        {\tt\small \{jezekp, moucek\}@kiv.zcu.cz}}%
\thanks{*This publication was supported by the project LO1506 of the Czech Ministry of Education, Youth and Sports}% <-this % stops a space
}
\maketitle



\begin{abstract}
In these days the management of diseases is still more expensive. For a long time people have been treated in hospitals. Fortunately the current situation is changing also thanks to relatively cheap body sensors and raising development of systems for home treatment. It brings inconsiderable costs saving and improves patients comfort. On the other hand it puts demands on sensors and home treatment system developers. They must solve integration of different systems. The crucial point is definition of unified data formats that facilitates data transfer and storage in remote databases. There are standards and APIs such as Zigbee, Bluetooth low energy or ANT+ that define a protocol for data transfer. However, they do not define format for data storing. As a solution data coming from ANT+ sensors have been studied. Then, metadata common to all sensors and raw data differing for each sensor have been defined. Next, framework for transferring data from ANT+ sensors into the proposed format implemented in NIX is described. Last, a use-case describing the transfer of data from a selected sensor into a selected neuroinformatics database is described.



\end{abstract}

\begin{IEEEkeywords}
ANT+, NIX, sensor, data/metadata storage, EEGBase, data format, eHealth.
\end{IEEEkeywords}




\section{Introduction}\label{sec:intro}
Management of diseases and health issues are still more expensive worldwide, especially with aging population. For instance, there were around 23 million people affected with heart failure in~\cite{bui2011epidemiology}. These people had been treated in hospitals for a long time. Nowadays, the situation is changing because relatively cheap solutions for home treatment are coming to the market~\cite{4761985, 5333913} and patients are shifted from hospitals to their homes. It brings advantages in a better comfort for patients and makes treatment cheaper. Home treatment systems use a set of wearable sensors, usually powered from batteries, for monitoring of health or fitness level. They have to operate for a long time period without possibility to change batteries frequently. That is why new protocols with low energy consumption as ZigBee~\cite{Farahani:2008:ZWN:1457417}, Bluetooth Low Energy~\cite{heydon2012bluetooth} or ANT~\cite{zaloker2014ant} have been developed. Data from these sensors are transferred to remote servers where they are processed and visualized. Body Area Networks (BAN) is an integration of sensors providing a~large data collection of body parameters.  When the number of sensors connected to BAN increases, requirements for the management, long term storage and sustainability of acquired data also increases.

Although there are some low energy consumption standards for data transfer, these are too fragmented to allow easy manipulation with obtained data. Of course, these standards also do not provide means for long term storage and management of transferred data. As a solution this paper presents how to use a~general data format called NIX~\cite{Stoewer:2014} for encapsulating and storing ANT+ sensor data.

The paper is organized as follows. Section~\ref{sec:state-of-the-art} deals with sensor infrastructure and description of data from obtained from sensors. Section~\ref{sec:ant-plus-profiles} describes existing ANT+ profiles; most suitable profiles for eHealth domain are selected. Section~\ref{sec:framework} introduces a~framework that facilitates conversion of sensor data to the NIX format. Section~\ref{sec:use-case} presents the usage of proposed transformation, a~simple use-case is provided. Section~\ref{sec:future-work} summarizes the work and provides an outlook to the future.

\section{State of The Art}\label{sec:state-of-the-art}

Three layers ontology describing data from different sensors is presented in~\cite{mehmood2014ontology}. This ontology facilitates development of tools for processing sensor data. A~Sensor-Cloud infrastructure~\cite{5635688} represents physical sensors as virtual sensors stored in a~Cloud infrastructure. The Cloud manages sensors capabilities. So called Semantic Sensor Web \cite{4557983} is based on annotation of sensors data by means of Semantic Web. Such annotated data can be distributed via the Internet.


\section{ANT+ Profiles Discussion}\label{sec:ant-plus-profiles}
ANT+ protocol is popular because of its low energy consumption and existence of suitable means for body parameters description. The main benefit are 'device profiles' that define sending data over the network in a consistent way~\cite{innovations2013ant}. Profiles facilitate development of sensors and management of sensor data in application programs.

ANT+ profiles support a large scale of activities such as cycling, walking, or measurement of body parameters such as heart rate, blood pressure, weight, or muscle oxygen. When browsing individual profiles in a detail we find the attributes that are common for all profiles, for example a device name, device status, manufacturer, signal strength or battery status. Then, there are attributes varying for individual profiles. Individual profile attributes represent raw sensor data or domain specific metadata while common attributes describe general metadata (see Figure~\ref{AntPlus}).


\begin{figure*}
\centering\includegraphics[width=12cm, height=10cm]{AntPlusProfiles}
\caption{\label{AntPlus}ANT+ Profiles Network}
\end{figure*}


\section{Proposed Framework}\label{sec:framework}

\subsection{Prerequisites}\label{sec:requirements}

Due to the absence of a suitable format/data structure for sensor data representation we have designed a framework for collection and storage of sensor raw data and metadata in a~defined structure. The used data structure/format has to robust, flexible and widely accepted by scientific community to cover a~heterogeneous nature of sensor data, provide a long term data sustainability, and ensure its re-usability in third-party systems.

\subsection{Format Discussion}

Within a working group of International Neuroinformatics Coordinating Facility (INCF)~\cite{wvangeit:Bjaalie:JNeurosci:2007} and its Task Force on Electrophysiology\footnote{http://www.incf.org/programs/datasharing/ electrophysiology-task-force} there were introduced two approaches towards defining a~standard on electrophysiology data. The first one uses the Hierarchical Data Format (HDF5)~\cite{hdf5}. HDF5 is portable and extensible format supporting an unlimited variety of datatypes that is designed for flexible and efficient I/O operations with high volume and complex data. The second approach uses odML~\cite{10.3389/fninf.2011.00016} as a~free form tree-like structure of sections, properties and values suitable for metadata description. This simple, platform-independent and human-readable format also ensures compatibility with other systems developed within the community such as~\cite{10.3389/conf.fninf.2014.18.00029},~\cite{10.3389/conf.fninf.2014.18.00053},~and~\cite{10.3389/conf.fninf.2013.09.00025}. The next step of the task force, merge of these two approaches~\cite{10.3389/conf.fninf.2013.09.00069}, resulted in the proposal of the NIX format~\cite{Stoewer:2014} that provides a~data model for storing experimental data in HDF5 together with its metadata internally organized in the odML format. The NIX format is currently used in Helmholtz~\cite{10.3389/conf.fninf.2013.09.00025} and EEGBase~\cite{ISI:000306821100004} projects. 

Although the NIX format was intended to be used in electrophysiology, its general definition makes it suitable for any time series data.


\subsection{Proposed Mapping}

We selected ANT+ profiles relating to person health or fitness level. Figure~\ref{AntPlus} shows common metadata (see the central circle) and domain specific raw data and metadata (see other circles). The NIX model consists of several main elements: Block, DataArray, Tag, MultiTag, Source, Group, and Dimension. Each element includes a~set of attributes (such as id, name, specific attributes) and link to metadata organized in the odML structure. We used a~simplified NIX model for mapping ANT+ elements. The Source element represents an ANT+ device, DataArray represents raw data, Dimension represents description of graph axes, and Block wraps a~complete record.

\begin{figure}
  %\vspace{-0.2cm}
  \centering\includegraphics[width=8cm]{portal_example.png}
  \caption{Metadata stored in EEGBase}
  \label{fig:EEGBase}
 \end{figure}


\section{Use Case}\label{sec:use-case}

The presented framework serves mainly to designers and programmers of the systems for home monitoring. Let's assume the following use-case. A programmer wants to implement a system for heart rate monitoring of elderly people. There are the following system requirements: sensors have to be easy to use and  sensors data must be easily transferred to a computer where they are stored and evaluated. Moreover, during regular medical checks a~physician uses long term records to check health condition of the patient or eventually starts a treatment. This means that both the patient and physician have access to the infrastructure that ensures data storing and management as well as data security, consistency and sustainability.

In this use-case we used the Garmin Premium Strap Heart Rate Monitor as a representative of ANT+ supporting devices. The Android SDK\footnote{http://developer.android.com/sdk/} was used to read ANT+ data at Android smart phones. We integrated this SDK into a custom mobile application MoBio\footnote{https://github.com/NEUROINFORMATICS-GROUP-FAV-KIV-ZCU/MoBio} that reads data from ANT+ sensors and store them on a~SD card. The user of MoBio can pair available sensors, record data and visualize them. This solution is available to a large number of users due to the existence of cheap Android smart phones and heart rate monitor straps on the market.

Since our framework was also integrated into MoBio, recorded data can be stored in the NIX format. MoBio parses the record, metadata are transferred into a~structure with one section and several properties(see an example in Figure~\ref{odML}) and continuously read heart beats data are stored into the DataArray element (see an example in Figure~\ref{NIX-ex}). The Source element has an attribute metadata that contains a~link to the odML structure.

Once the data and metadata are stored they can be transferred to a suitable database. Figure~\ref{fig:EEGBase} shows the metadata stored and visualized in EEGBase. A~complete description of the experiment contains metadata from the Heart Rate strap. The raw data are stored as well.

\begin{figure}

\dirtree{%
.1 Section.
.2 name = heart\_rate.
.2 type = metadata.
.2 Properties.
.3 Device name = Strap heart rate monitor.
.3 Device number = 1.
.3 Product Information = Garmin Premium Strap Heart Rate Monitor.
}
\caption{\label{odML}Metadata from heart rate sensor in odML structure}
\end{figure}

\begin{figure*}
\centering\includegraphics[width=11cm]{NIX-example.eps}
\caption{\label{NIX-ex}Heart rate record in the NIX format}
\end{figure*}



\section{Conclusions and Future Work}\label{sec:future-work}

Because of raising popularity of sensors for home treatment several low energy standards has been defined. These standards enable data transfer from body sensors into common computers where they can be processed. The one of the most often used is ANT+ supported by significant sensors producers. Although a transfer protocol is defined and several APIs for working with sensors exist there is not defined a standard for storing of sensors data. As long as home treatment systems use proprietary data formats systems can not be easily integrated with sensors. It brings serious complications for designers of such systems. In this paper we described a framework that overcomes these difficulties by designing a framework that maps data from ANT+ sensors into a widely accepted format NIX. This format brings advantages of two layers structure. First the metadata structure uses a flexible format odML and the second data structure is based on HDF5 format. The definition of this two layers of ANT+ sensors is also significant contribution of this work.

The functionality of the framework is described on a simple use case. Our future work is to test it on a large collection of sensors and test the data transfer to a larger collection of neuroinformatics databases. We also plan to invite developers of home treatment systems to integrate the framework into their solutions.

Next, when the framework is fully tested we will work on the transformation of data from Blueetooth low energy into the NIX format as well.



% Now here is the reference section.

\bibliographystyle{IEEEtran}
% argument is your BibTeX string definitions and bibliography database(s)
\bibliography{EMBC-2016,citations-ic4awe-2016,bibliography,neuroportals}
\end{document} 